%%%%%%%%%%%%%%%%%%%%%%%%%%%%%%%%%%%%%%%%%
% Structured General Purpose Assignment
% LaTeX Template
%
% This template has been downloaded from:
% http://www.latextemplates.com
%
% Original author:
% Ted Pavlic (http://www.tedpavlic.com)
%
% Note:
% The \lipsum[#] commands throughout this template generate dummy text
% to fill the template out. These commands should all be removed when 
% writing assignment content.
%
%%%%%%%%%%%%%%%%%%%%%%%%%%%%%%%%%%%%%%%%%

%----------------------------------------------------------------------------------------
%	PACKAGES AND OTHER DOCUMENT CONFIGURATIONS
%----------------------------------------------------------------------------------------

\documentclass{article}

\usepackage{fancyhdr} % Required for custom headers
\usepackage{lastpage} % Required to determine the last page for the footer
\usepackage{extramarks} % Required for headers and footers
\usepackage{graphicx} % Required to insert images
\usepackage{lipsum} % Used for inserting dummy 'Lorem ipsum' text into the template
\usepackage{listings}
\usepackage{color}
\usepackage{amsmath}

\definecolor{dkgreen}{rgb}{0,0.6,0}
\definecolor{gray}{rgb}{0.5,0.5,0.5}
\definecolor{mauve}{rgb}{0.58,0,0.82}

\lstset{frame=tb,
  language=Python,
  aboveskip=3mm,
  belowskip=3mm,
  showstringspaces=false,
  columns=flexible,
  basicstyle={\small\ttfamily},
  numbers=none,
  numberstyle=\tiny\color{gray},
  keywordstyle=\color{blue},
  commentstyle=\color{dkgreen},
  stringstyle=\color{mauve},
  breaklines=true,
  breakatwhitespace=true
  tabsize=3
}

% Margins
\topmargin=-0.45in
\evensidemargin=0in
\oddsidemargin=0in
\textwidth=6.5in
\textheight=9.0in
\headsep=0.25in 

\linespread{1.1} % Line spacing

% Set up the header and footer
\pagestyle{fancy}
\lhead{\hmwkAuthorName} % Top left header
\chead{\hmwkClass\ (\hmwkClassInstructor\ \hmwkClassTime): \hmwkTitle} % Top center header
\rhead{\firstxmark} % Top right header
\lfoot{\lastxmark} % Bottom left footer
\cfoot{} % Bottom center footer
\rfoot{Page\ \thepage\ of\ \pageref{LastPage}} % Bottom right footer
\renewcommand\headrulewidth{0.4pt} % Size of the header rule
\renewcommand\footrulewidth{0.4pt} % Size of the footer rule

\setlength\parindent{0pt} % Removes all indentation from paragraphs

%----------------------------------------------------------------------------------------
%	DOCUMENT STRUCTURE COMMANDS
%	Skip this unless you know what you're doing
%----------------------------------------------------------------------------------------

% Header and footer for when a page split occurs within a problem environment
\newcommand{\enterProblemHeader}[1]{
\nobreak\extramarks{#1}{#1 continued on next page\ldots}\nobreak
\nobreak\extramarks{#1 (continued)}{#1 continued on next page\ldots}\nobreak
}

% Header and footer for when a page split occurs between problem environments
\newcommand{\exitProblemHeader}[1]{
\nobreak\extramarks{#1 (continued)}{#1 continued on next page\ldots}\nobreak
\nobreak\extramarks{#1}{}\nobreak
}

\setcounter{secnumdepth}{0} % Removes default section numbers
\newcounter{homeworkProblemCounter} % Creates a counter to keep track of the number of problems

\newcommand{\homeworkProblemName}{}
\newenvironment{homeworkProblem}[1][Problem \arabic{homeworkProblemCounter}]{ % Makes a new environment called homeworkProblem which takes 1 argument (custom name) but the default is "Problem #"
\stepcounter{homeworkProblemCounter} % Increase counter for number of problems
\renewcommand{\homeworkProblemName}{#1} % Assign \homeworkProblemName the name of the problem
\section{\homeworkProblemName} % Make a section in the document with the custom problem count
\enterProblemHeader{\homeworkProblemName} % Header and footer within the environment
}{
\exitProblemHeader{\homeworkProblemName} % Header and footer after the environment
}

\newcommand{\problemAnswer}[1]{ % Defines the problem answer command with the content as the only argument
\noindent\framebox[\columnwidth][c]{\begin{minipage}{0.98\columnwidth}#1\end{minipage}} % Makes the box around the problem answer and puts the content inside
}

\newcommand{\homeworkSectionName}{}
\newenvironment{homeworkSection}[1]{ % New environment for sections within homework problems, takes 1 argument - the name of the section
\renewcommand{\homeworkSectionName}{#1} % Assign \homeworkSectionName to the name of the section from the environment argument
\subsection{\homeworkSectionName} % Make a subsection with the custom name of the subsection
\enterProblemHeader{\homeworkProblemName\ [\homeworkSectionName]} % Header and footer within the environment
}{
\enterProblemHeader{\homeworkProblemName} % Header and footer after the environment
}
   
%----------------------------------------------------------------------------------------
%	NAME AND CLASS SECTION
%----------------------------------------------------------------------------------------

\newcommand{\hmwkTitle}{Quiz\ 3} % Assignment title
\newcommand{\hmwkDueDate}{July 17,\ 2014} % Due date
\newcommand{\hmwkClass}{Advanced Calculus with FE Application} % Course/class
\newcommand{\hmwkClassTime}{} % Class/lecture time
\newcommand{\hmwkClassInstructor}{Advisor: Dan Stefanica} % Teacher/lecturer
\newcommand{\hmwkAuthorName}{Weiyi Chen} % Your name

%----------------------------------------------------------------------------------------
%	TITLE PAGE
%----------------------------------------------------------------------------------------

\title{
\vspace{2in}
\textmd{\textbf{\hmwkClass:\ \hmwkTitle}}\\
\normalsize\vspace{0.1in}\small{Due\ on\ \hmwkDueDate}\\
\vspace{0.1in}\large{\textit{\hmwkClassInstructor\ \hmwkClassTime}}
\vspace{3in}
}

\author{\textbf{\hmwkAuthorName}}
\date{} % Insert date here if you want it to appear below your name

%----------------------------------------------------------------------------------------

\begin{document}

\maketitle

%----------------------------------------------------------------------------------------
%	TABLE OF CONTENTS
%----------------------------------------------------------------------------------------

%\setcounter{tocdepth}{1} % Uncomment this line if you don't want subsections listed in the ToC

%\newpage
%\tableofcontents
\newpage

%----------------------------------------------------------------------------------------
%	PROBLEM 1
%----------------------------------------------------------------------------------------

\begin{homeworkProblem}
    \begin{homeworkSection}{(i)}
        Since the covariance of $Z_1$ and $Z_2$ is $0.3$, then
        \begin{equation}
            \begin{split}
                X &= 3Z_1 - Z_2 \\
                &= (3\mu_1 - \mu_2) + \sqrt{(3\sigma_1)^2 + \sigma_2^2 - 2cov(3Z_1, Z_2)}Z \\
                &= (3\times0-0) + \sqrt{(3^2+1^2-6\times0.3)} \\
                &= \sqrt{8.2}Z
            \end{split}
        \end{equation}
        Therefore the mean and the variance of $X$ is
        \begin{equation}
             \begin{split}
                E[X] &= 0 \\
                var[X] &= 8.2
             \end{split}
        \end{equation} 
    \end{homeworkSection}
    \begin{homeworkSection}{(ii)}
        If $Z_1$ and $Z_2$ are independent, then
        \begin{equation}
            cov(Z_1, Z_2) = 0
        \end{equation}
        Using the similar process in part (i), we will have 
        \begin{equation}
            X = \sqrt{10}Z
        \end{equation}
        Therefore the probability density function of $X$ is
        \begin{equation}
            \begin{split}
                f_X(x) &= \frac{1}{\sigma\sqrt{2\pi}} e^{-\frac{(x-\mu)^2}{2\sigma^2}} \\
                &= \frac{1}{\sqrt{20\pi}} e^{-\frac{x^2}{20}}
            \end{split}
        \end{equation}
        where $\mu = 0, \sigma = \sqrt{10}$.
    \end{homeworkSection}
\end{homeworkProblem}
\newpage

%----------------------------------------------------------------------------------------
%   PROBLEM 2
%----------------------------------------------------------------------------------------

\begin{homeworkProblem}
    The probability that $X$ is greater than 2 is
    \begin{equation}
        \begin{split}
            Pr(X>2) &= Pr(3+Z>2) \\
            &= Pr(Z>-1) \\
            &= Pr(Z<1) \\
            &= N(1) \\
            &= 0.8413
        \end{split}
    \end{equation}
    where $N(1) = 0.8413$ is given and $N(t)$ is the cumulative density of the standard normal variable.
\end{homeworkProblem}
\newpage

%----------------------------------------------------------------------------------------
%   PROBLEM 3
%----------------------------------------------------------------------------------------

\begin{homeworkProblem}
    \begin{homeworkSection}{Quiz's formula for CoN put}
        In the quiz, the value of cash-or-nothing put is defined as
        \begin{equation}
            P_{CoN} = Be^{-qT}N(-d_1)
        \end{equation}
        Delta of the cash-or-nothing put is 
        \begin{equation}
            \begin{split}
                \Delta(P_{CoN}) &= \frac{\partial P_{CoN}}{\partial S} \\
                &= Be^{-qT} N'(-d_1) (-\frac{\partial d_1}{\partial S}) \\
                &= Be^{-qT} (\frac{1}{\sqrt{2\pi}}e^{-d_1^2/2}) (-\frac{1}{\sigma\sqrt{T}} \frac{K}{S} \frac{1}{K}) \\
                &= -\frac{Be^{-qT-d_1^2/2}}{\sigma\sqrt{2\pi T}S}
            \end{split}
        \end{equation}
        where 
        \begin{equation}
            d_1 = \frac{\ln(\frac{S}{K}) + (r-q+\frac{\sigma^2}{2})T}{\sigma\sqrt{T}}
        \end{equation}
    \end{homeworkSection}
    \begin{homeworkSection}{Correct formula for CoN put}
        However, TA pointed out in the forum saying the value of cash-or-nothing put should be
        \begin{equation}
            P_{CoN} = Be^{-rT}N(-d_2)
        \end{equation}
        Then in similar way, delta of the put is
        \begin{equation}
            \Delta(P_{CoN}) = -\frac{Be^{-rT-d_2^2/2}}{\sigma\sqrt{2\pi T}S}
        \end{equation}
        where
        \begin{equation}
            d_2 = \frac{\ln(\frac{S}{K}) + (r-q-\frac{\sigma^2}{2})T}{\sigma\sqrt{T}}
        \end{equation}
    \end{homeworkSection}
\end{homeworkProblem}
\newpage

%----------------------------------------------------------------------------------------
%   PROBLEM 4
%----------------------------------------------------------------------------------------

\begin{homeworkProblem}
    Using the Black-Scholes formula with input 
    \begin{equation}
        \begin{split}
            S_1 &= K = 100 \\
            T &= 1/2 \\
            \sigma &= 0.3 \\
            r &= 0.05 \\
            q &= 0 
        \end{split}
    \end{equation}
    we find that the value of one put options is
    \begin{equation}
        P_1 = 7.16586783128
    \end{equation}
    Therefore, $\$7,165.86143674$ must be paid for 1000 puts.
    \begin{homeworkSection}{(i)}
        The Delta-hedging position for long 1000 puts is short
        \begin{equation}
            1000 \Delta(P) = 1000(-e^{-qT}N(-d_1)) = -411.41
        \end{equation}
        units of the underlying. Therefore, $411$ units of the underlying must be longed. I will have to borrow \$41,100 to buy the underlying.
    \end{homeworkSection}
    \begin{homeworkSection}{(ii)}
        The new spot price and muturity of the option are 
        \begin{equation}
            S_2 = 102, T_2 = 125/252
        \end{equation}
        since there are 252 trading days in one year. The value of the put option is
        \begin{equation}
            P_2 = 6.35543075983
        \end{equation}
        and the value of the portfolio is 
        \begin{equation}
            1000P_2 + 411S_2 - 411S_1e^{\frac{r}{252}} = 7169.27518887
        \end{equation}
    \end{homeworkSection}
    \begin{homeworkSection}{(iii)}
        If the long put position is not Delta-hedged, the loss incurred due to the increase in the spot price of the underlying asset is 
        \begin{equation}
            1000(P_2 - P_1) = -\$810.43707145
        \end{equation}
        For the Delta-hedged portfolio, the loss incurred is 
        \begin{equation}
            (1000P_2+411S_2-411S_1e^{\frac{r}{252}}) - (1000P_1+411S_1-411S_1) = \$3.40735758799
        \end{equation}
        As expected, this loss is much smaller than the loss incurred if the options positions is not hedged.
    \end{homeworkSection}
\end{homeworkProblem}

\end{document}