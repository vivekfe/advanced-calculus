%%%%%%%%%%%%%%%%%%%%%%%%%%%%%%%%%%%%%%%%%
% Structured General Purpose Assignment
% LaTeX Template
%
% This template has been downloaded from:
% http://www.latextemplates.com
%
% Original author:
% Ted Pavlic (http://www.tedpavlic.com)
%
% Note:
% The \lipsum[#] commands throughout this template generate dummy text
% to fill the template out. These commands should all be removed when 
% writing assignment content.
%
%%%%%%%%%%%%%%%%%%%%%%%%%%%%%%%%%%%%%%%%%

%----------------------------------------------------------------------------------------
%	PACKAGES AND OTHER DOCUMENT CONFIGURATIONS
%----------------------------------------------------------------------------------------

\documentclass{article}

\usepackage{fancyhdr} % Required for custom headers
\usepackage{lastpage} % Required to determine the last page for the footer
\usepackage{extramarks} % Required for headers and footers
\usepackage{graphicx} % Required to insert images
\usepackage{lipsum} % Used for inserting dummy 'Lorem ipsum' text into the template
\usepackage{listings}
\usepackage{color}
\usepackage{amsmath}

\definecolor{dkgreen}{rgb}{0,0.6,0}
\definecolor{gray}{rgb}{0.5,0.5,0.5}
\definecolor{mauve}{rgb}{0.58,0,0.82}

\lstset{frame=tb,
  language=R,
  aboveskip=3mm,
  belowskip=3mm,
  showstringspaces=false,
  columns=flexible,
  basicstyle={\small\ttfamily},
  numbers=none,
  numberstyle=\tiny\color{gray},
  keywordstyle=\color{blue},
  commentstyle=\color{dkgreen},
  stringstyle=\color{mauve},
  breaklines=true,
  breakatwhitespace=true
  tabsize=3
}

% Margins
\topmargin=-0.45in
\evensidemargin=0in
\oddsidemargin=0in
\textwidth=6.5in
\textheight=9.0in
\headsep=0.25in 

\linespread{1.1} % Line spacing

% Set up the header and footer
\pagestyle{fancy}
\lhead{\hmwkAuthorName} % Top left header
\chead{\hmwkClass\ (\hmwkClassInstructor\ \hmwkClassTime): \hmwkTitle} % Top center header
\rhead{\firstxmark} % Top right header
\lfoot{\lastxmark} % Bottom left footer
\cfoot{} % Bottom center footer
\rfoot{Page\ \thepage\ of\ \pageref{LastPage}} % Bottom right footer
\renewcommand\headrulewidth{0.4pt} % Size of the header rule
\renewcommand\footrulewidth{0.4pt} % Size of the footer rule

\setlength\parindent{0pt} % Removes all indentation from paragraphs

%----------------------------------------------------------------------------------------
%	DOCUMENT STRUCTURE COMMANDS
%	Skip this unless you know what you're doing
%----------------------------------------------------------------------------------------

% Header and footer for when a page split occurs within a problem environment
\newcommand{\enterProblemHeader}[1]{
\nobreak\extramarks{#1}{#1 continued on next page\ldots}\nobreak
\nobreak\extramarks{#1 (continued)}{#1 continued on next page\ldots}\nobreak
}

% Header and footer for when a page split occurs between problem environments
\newcommand{\exitProblemHeader}[1]{
\nobreak\extramarks{#1 (continued)}{#1 continued on next page\ldots}\nobreak
\nobreak\extramarks{#1}{}\nobreak
}

\setcounter{secnumdepth}{0} % Removes default section numbers
\newcounter{homeworkProblemCounter} % Creates a counter to keep track of the number of problems

\newcommand{\homeworkProblemName}{}
\newenvironment{homeworkProblem}[1][Problem \arabic{homeworkProblemCounter}]{ % Makes a new environment called homeworkProblem which takes 1 argument (custom name) but the default is "Problem #"
\stepcounter{homeworkProblemCounter} % Increase counter for number of problems
\renewcommand{\homeworkProblemName}{#1} % Assign \homeworkProblemName the name of the problem
\section{\homeworkProblemName} % Make a section in the document with the custom problem count
\enterProblemHeader{\homeworkProblemName} % Header and footer within the environment
}{
\exitProblemHeader{\homeworkProblemName} % Header and footer after the environment
}

\newcommand{\problemAnswer}[1]{ % Defines the problem answer command with the content as the only argument
\noindent\framebox[\columnwidth][c]{\begin{minipage}{0.98\columnwidth}#1\end{minipage}} % Makes the box around the problem answer and puts the content inside
}

\newcommand{\homeworkSectionName}{}
\newenvironment{homeworkSection}[1]{ % New environment for sections within homework problems, takes 1 argument - the name of the section
\renewcommand{\homeworkSectionName}{#1} % Assign \homeworkSectionName to the name of the section from the environment argument
\subsection{\homeworkSectionName} % Make a subsection with the custom name of the subsection
\enterProblemHeader{\homeworkProblemName\ [\homeworkSectionName]} % Header and footer within the environment
}{
\enterProblemHeader{\homeworkProblemName} % Header and footer after the environment
}
   
%----------------------------------------------------------------------------------------
%	NAME AND CLASS SECTION
%----------------------------------------------------------------------------------------

\newcommand{\hmwkTitle}{Quiz 4} % Assignment title
\newcommand{\hmwkDueDate}{July 22,\ 2014} % Due date
\newcommand{\hmwkClass}{Advanced Calculus with FE Application} % Course/class
\newcommand{\hmwkClassTime}{} % Class/lecture time
\newcommand{\hmwkClassInstructor}{Dan Stefanica} % Teacher/lecturer
\newcommand{\hmwkAuthorName}{Weiyi Chen} % Your name

%----------------------------------------------------------------------------------------
%	TITLE PAGE
%----------------------------------------------------------------------------------------

\title{
\vspace{2in}
\textmd{\textbf{\hmwkClass:\ \hmwkTitle}}\\
\normalsize\vspace{0.1in}\small{Due\ on\ \hmwkDueDate}\\
\vspace{0.1in}\large{\textit{\hmwkClassInstructor\ \hmwkClassTime}}
\vspace{3in}
}

\author{\textbf{\hmwkAuthorName}}
\date{} % Insert date here if you want it to appear below your name

%----------------------------------------------------------------------------------------

\begin{document}

\maketitle

%----------------------------------------------------------------------------------------
%	TABLE OF CONTENTS
%----------------------------------------------------------------------------------------

%\setcounter{tocdepth}{1} % Uncomment this line if you don't want subsections listed in the ToC

%\newpage
%\tableofcontents
\newpage

%----------------------------------------------------------------------------------------
%	PROBLEM 1
%----------------------------------------------------------------------------------------

\begin{homeworkProblem}
    \begin{homeworkSection}{(i)}
        Binomial series (includes the square root for $\alpha = 0.5$)
        \begin{equation}
            (1+x)^\alpha = \sum_{n=0}^\infty {\alpha \choose n} x^n\quad\text{ for all }|x| < 1 \text{ and all complex } \alpha\!
        \end{equation}
        with generalized binomial coefficients
        \begin{equation}
            {\alpha\choose n} = \prod_{k=1}^n \frac{\alpha-k+1}k = \frac{\alpha(\alpha-1)\cdots(\alpha-n+1)}{n!}.
        \end{equation}
        For $\alpha = 0.5$,
        \begin{equation}
            \sqrt{1+x} = \sum_{n=0}^\infty {0.5 \choose n} x^n\quad\text{ for all }|x| < 1
        \end{equation}
        Or
        \begin{equation}
            \sqrt{1+x} = \textstyle 1 + \frac{1}{2}x - \frac{1}{8}x^2 + \frac{1}{16}x^3 - \frac{5}{128}x^4 + \frac{7}{256}x^5 + O(x^6)
        \end{equation}
        which converges when $|x| < 1$. \\
        Substituting $x$ with $-x^4$, the taylor series expansion of $f_1(x)$ around the point 0 is
        \begin{equation}
            \begin{split}
                f_1(x) &= \sum_{n=0}^{\infty} \frac{f_1^{(n)}(0)}{n!} x^{n} \\
                &= 1-\frac{x^4}{2}-\frac{x^8}{8}-\frac{x^{12}}{16}-\frac{5x^{16}}{128}-\frac{7x^{20}}{256}-\frac{21x^{24}}{1024}+O(x^{25}) \\
                &= \sum_{n = 0}^{\infty} (-1)^n {0.5 \choose n} x^{4n} \\
                &= \sum_{n = 0}^{\infty} \frac{(2n)!}{(1-2n)(n!)^2(4^n)} x^{4n}
            \end{split}
        \end{equation}
        which converges when $|-x^4| < 1$, or $|x| < 1$. We can also use the formula to calculate, where using stirling's formula
        \begin{equation}
            a_n = (-1)^n {0.5 \choose n} \approx -\frac{(-1.5+n)!}{2\sqrt{\pi}n!}
        \end{equation}
        The radius of convergence is 
        \begin{equation}
            \begin{split}
                R &= \frac{1}{\lim\sup_{n\to\infty}|a_n|^{1/n}} \\
                &= \frac{1}{\lim\sup_{n\to\infty}|-\frac{(-1.5+n)!}{2\sqrt{\pi}n!}|^{1/n}} \\
                &= 1                
            \end{split}
        \end{equation}
    \end{homeworkSection}
    \begin{homeworkSection}{(ii)}
        Exponential function
        \begin{equation}
            e^{-x} = \sum^{\infty}_{n=0} \frac{(-x)^n}{n!} = 1 - x + \frac{x^2}{2!} - \frac{x^3}{3!} + O(x^4) \quad\text{ for all } x\!
        \end{equation}
        which converges everywhere. \\
        Substitute $x$ with $x^2$, the taylor series expansion of $e^{-x^2}$ around the point 0 is
        \begin{equation}
            e^{-x^2} = \sum^{\infty}_{n=0} \frac{(-x^2)^n}{n!}
        \end{equation}
        which still converges everywhere. \\
        Therefore the taylor series expansion of $f_2(x)$ around the point 0 is
        \begin{equation}
            \begin{split}
                f_2(x) &= e^2 e^{-x^2} \\
                &= \sum^{\infty}_{n=0} \frac{e^2(-x^2)^n}{n!} \\
                &= \sum^{\infty}_{n=0} \frac{e^2(-1)^n}{n!} x^{2n}
            \end{split}
        \end{equation}
        which converges everywhere, that is, with the radius of convergence as $R = \infty$.
    \end{homeworkSection}
\end{homeworkProblem}

%----------------------------------------------------------------------------------------
%   PROBLEM 2
%----------------------------------------------------------------------------------------

\begin{homeworkProblem}
    \begin{homeworkSection}{(i)}
        The dollar duration of the portfolio is
        \begin{equation}
            D_{\$} = \sum_i D_iB_i = 20 \times 3 + 50 \times 4 = 260
        \end{equation}
        The dollar convexity of the portfolio is 
        \begin{equation}
            C_{\$} = \sum_i C_iB_i = 20 \times 18 + 50 \times 20 = 1360
        \end{equation}
    \end{homeworkSection}
    \begin{homeworkSection}{(ii)}
        The change of portfolio value is
        \begin{equation}
            \begin{split}
                \Delta B &= -D_{\$}\Delta y + \frac{1}{2} C_{\$}(\Delta y)^2  \\
                &= -260\times0.25\% + 0.5\times1360\times(0.25\%)^2 \\
                &= -0.64575
            \end{split}
        \end{equation}
        The approximate value of the portfolio is
        \begin{equation}
            B_{new} = B_{old} + \Delta B = 69.35425
        \end{equation}
        million dollars.
    \end{homeworkSection}
    \begin{homeworkSection}{(iii)}
        Suppose we long $x_1$ units of the bond with duration 2 and convexity 7, and long $x_2$ units of the bond with duration 4 and convexity 11, to  immunize the portfolio, then
        \begin{equation}
            \begin{split}
                D_{\$} &= \sum_i D_iB_i = 260 + 2x_1 + 4x_2 = 0 \\
                C_{\$} &= \sum_i D_iC_i = 1360 + 7x_1 + 11x_2 = 0
            \end{split}
        \end{equation}
        which solved as $x_1=-430, x_2=150$, that is
        \begin{itemize}
            \item Short $430$ units of the bond with duration 2 and convexity 7
            \item Long $150$ units of the bond with duration 4 and convexity 11
        \end{itemize}
    \end{homeworkSection}
\end{homeworkProblem}

\end{document}